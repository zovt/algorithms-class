\documentclass[14pt, letterpaper]{article}
\usepackage[utf8]{inputenc}
\usepackage{parskip}
\usepackage{listings}
\usepackage{amsmath}

\title{Algorithms and Data --- Problem Set 1}
\author{Nick Ippoliti}
\date{September 20, 2016}

\begin{document}
\begin{titlepage}
	\maketitle
\end{titlepage}

\section{Three Arrays}
\subsection{a.}
A brute force algorithm would be simply to iterate over each element $a_i$ in 
list $a$ and pair each of those with every element $b_j$ in list $b$. Then it
would be necessay for each pair $(a_i, b_j)$ to iterate over each element $c_k$
in list $c$ to see if $a_i + b_j = c_k$.

The running time of this algorithm would be $O(n^3)$.

\subsection{b.}
A smarter algorithm would be to sort column $c$ from smallest to largest. Then,
iterate over each element $a_i$ in list $a$ and sum it with each element $b_j$
in list $b$. Then take that sum and perform a binary search for it in list $c$.
If it exists, return true. Otherwise, do the same for the next sum until out of
pairs.

This running time would be $O(n^2 \log n)$. The sort would take $n \log n$
time, but that is only performed once. The iteration and lookup step would take
$n^2 \log n$ time since a binary search is $O(\log n)$.

\section{}
\subsection{a.}
A brute force algorithm would be to iterate over each element $a_i$ in list $a$
and for each of the remaining elements $a_j, j \neq i$, calculate the difference
between them. If that difference $\Delta_{new}$ is less than the current 
smallest difference $\Delta_{current}$, set $\Delta_{current} =  \Delta_{new}$.
Once that iteration has been completed for all $a_i$ in $a$, yield 
$\Delta_{current}$.

The running time of this algorithm would be O($n^2$)

\subsection{b.}
An efficient algorithm would be to first sort the elements from smallest to
largest, which would take $O(n \log n)$ time. Set $\Delta_{current} = \infty$.
Then, for each element $a_i$ in list $a$, calculate the difference
$\Delta_{new}$ between $a_i$ and $a_{i + 1}$. If
$\Delta_{new} < \Delta_{current}$, set $\Delta_{current} = \Delta_{new}$.
When this iteration is done, yield $\Delta_{current}$.

The running time of this would be $O(n \log n)$ time, since the iteration would
only take $O(n)$.

\section{Order of Growth}
Since $\displaystyle {n \choose k} = \Theta(n^k)$, $\displaystyle {3n \choose n} = \Theta((3n)^n)$

\section{Inequalities by Induction}
\subsection{a.}
$$
n^2 \geq 10n + 5 \\
(n + 1)^2 \geq 10(n + 1) + 5 \\
n^2 + 2n + 1 \geq 10(n + 1) + 5 \\
n^2 + 2n + 1 \geq 10(n + 1) + 5 + (10n + 5) \\
n^2 + 2n + 1 \geq 10n + 15 + (10n + 5) \\
n^2 + 2n + 1 \geq 20n + 20 \\
n^2 + 2n + 1 \geq 20n + 20 \\
n^2 + 2n + 1 \geq 20(n + 1) \\
(n+1)(n+1) \geq 20(n + 1) \\
n + 1 \geq 20 \\
n \geq 19 \\
$$

$n_0 = 19$

%TODO
\section{Matching the Bipartite Graphs}
$P(n):$

If $T_n = (U, F)$ is the minimum spanning tree of $G$ with $2n$ vertices and 
$M = (F_i) \forall i \in (1, 3, ..., |T_n|)$, then $M$ is a matching of $G$.

$|T_n| = 2n - 1$ \\
$|M| = \frac{|T_n| + 1}{2} = \frac{2n - 1 + 1}{2} = \frac{2n}{2} = n$ \\

Base Case: $P(n); n = 2$ \\
$|T_2| = 2(2) - 1 = 3$ \\
$|M| = \frac{|T_2| + 1}{2} = \frac{4}{2} = 2$ \\

Inductive case: \\
Assuming $P(n)$ holds, then

$|T_{n + 1}| = 2(n + 1) - 1 = 2n + 2 - 1 = 2n + 1$ \\
$|M| = \frac{|T_{n + 1}| + 1}{2} = \frac{(1 + |T_{n}| + 1)}{2} 
= \frac{2 + |T_n|}{2} = 1 + |T_n| = 1 + n$

\section{Ordered search with penalty}
A search method that is somewhere between these two methods would be Chunk
Search. The algorithm works by checking the first element $a_0$ in list $a$,
then jumping $\sqrt n$ indices, and checking that element $a_i$. This
repeats until a red element is found, in which case it jumps back to the last
green element and checks the elements one by one until it finds the index $m$
at which the elements transition from green to red.

This incurs at most \$2 of penalty, and is $O(\sqrt n)$ runtime.

\section{Growth of Functions}
\subsection{a.}
$f = \Omega(g)$, due to the fact that the power $n$ will grow faster than
$\log n$ as $n$ goes to $\infty$, and cause $f(n) \geq g(n) \forall n > n_0$.

\subsection{b.}
$f = O(g)$, since $\displaystyle \lim_{x \to \infty} \frac{\log^2 x}{^3\sqrt x} = 0$

\subsection{c.}
\begin{equation}
	\begin{split}
		f(n) = 100n + \frac{n^2 + 1}{\sqrt n} & = 100n + \frac{n^2}{\sqrt n} + \frac{1}{\sqrt n} \\
		& = 100n + n^{3/2} + 1/{\sqrt n} \\
		& = n^{3/2} + 100n + 1/{\sqrt n}
	\end{split}
\end{equation}
Therefore, $f = \Theta(g)$.

\end{document}
