\documentclass[14pt, letterpaper]{article}
\usepackage[utf8]{inputenc}
\usepackage{parskip}
\usepackage{algorithm}
\usepackage{listings}
\usepackage{amsmath}
\usepackage{amsfonts}

\title{Algorithms and Data --- Problem Set 3}
\author{Nick Ippoliti}
\date{October 25, 2016}

\begin{document}
\begin{titlepage}
\maketitle
\end{titlepage}

\section{The Interval Scheduling Again}
This approach is a greedy since it tries to make the locally optimal choice
with each interval selection, i.e., it picks the one with the last start time
that does not conflict with any of the others with the hope that by doing this
repeatedly we arrive at a globally optimal solution.

This does, in fact, yield an optimal solution. Let $I$ be a set of intervals.
We first pick an interval $I_1$ with start time $S_1$ such that $S_1$ is 
greater than all other start times. Then, we let $I'$ be a subset of $I$ such
that no intervals in $I'$ intersect with $I_1$. We repeat the process using 
$I'$, and continue repeating in this way until no intervals are left.

Thus, in essence, we are picking intervals that begin closest to the maximum
end time $E_{max}$ of $I$ and do not overlap any of our currently chosen 
intervals. The fact that we are choosing the interval with the start time 
closest to the maximum end time means that there can be other interval or
combination of intervals that do not intersect our current choices and minimize
the distance between their start time and the max end time. That is to say,
choosing in this way will give us the maximum number of non-ovelapping 
intervals in a given set of intervals.

\end{document}
