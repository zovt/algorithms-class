\documentclass[14pt, letterpaper]{article}
\usepackage[utf8]{inputenc}
\usepackage{parskip}
\usepackage{algorithm}
\usepackage{listings}
\usepackage{amsmath}
\usepackage{amsfonts}

\title{Algorithms and Data --- Problem Set 3}
\author{Nick Ippoliti}
\date{October 25, 2016}

\begin{document}
\begin{titlepage}
\maketitle
\end{titlepage}

\section{The Interval Scheduling Again}
This approach is a greedy since it tries to make the locally optimal choice
with each interval selection, i.e., it picks the one with the last start time
that does not conflict with any of the others with the hope that by doing this
repeatedly we arrive at a good global solution.

This does, in fact, yield an optimal solution. Let $I$ be a set of intervals.
We first pick an interval $I_1$ with start time $S_1$ such that $S_1$ is 
greater than all other start times. Then, we let $I'$ be a subset of $I$ such
that no intervals in $I'$ intersect with $I_1$. We repeat the process using 
$I'$, and continue repeating in this way until no intervals are left.

Thus, in essence, we are picking intervals that begin closest to the maximum
end time $E_{max}$ of $I$ and do not overlap any of our currently chosen 
intervals. The fact that we are choosing the interval with the start time 
closest to the maximum end time means that there can be no other interval or
combination of intervals that do not intersect our current choices and minimize
the distance between their start time and the max end time and that interval is
the smallest interval that is compatible with our choices. That is to say,
choosing in this way will give us a maximum number of non-ovelapping intervals 
in a given set of intervals.

\section{Connected Components}
Consider a graph with a single vertex and no edges. There is one connected
component: the node itself. Thus: $n - k = 1 - 1 = 0$.

Consider a graph with $n$ nodes and no edges, $e = 0$. This implies that there
are $k$ connected components currently and $k = n$. Drawing an edge between any
connected components, which at this point are the vertices, increases the edge
count $e'$ by 1 and reduces the number of connected components $k'$ by 1. This
means the if this was done $t$ times, we would now have $e' = t$ and 
$k' = n - t$. Simplifying, we get $e' = t = n - k'$. Since we can draw any
number of edges between vertices within the same connected component, $e'$
becomes an upper bound, $e' \geq n - k'$. Thus, the number of edges in a graph
$e$ will always be at least $n - k$.

\section{MST + Shortest Paths}
\subsection{a}
Replacing each edge cost by $\log(1 + c_e)$ will not change the minimum 
spanning tree. To see this, if we a have an ordered set of edge weights $E$,
mapping $\log(1 + E_i)$ over each weight to produce $E'$ will keep the ordering
of $E'$ and $E$ the same. That is, for every element $E'_i$ in $E'$,
$2^{E'_i} = E_i$. This is due to the fact that the log function is continuous
and increasing for every number greater than 0. Since the ordering of the
smallest edge weights would remain unchanged, the MST would remain the same.

\subsection{b}
Replacing each edge cost by $\sqrt{c_e}$ could change P. This is because square
root does not effect numbers linearly, and thus does not effect the sums of path
weights linearly. This would effect the path P, since finding the shortest path
would involve calculating the total edge weights. For example, if we had a path
from $u$ to $v$ containing edge weights 1, 1, and 1, and another path containing
edge weight 4, we can see that the first would be faster. However, computing the
square root would make the 4 into 2, causing the optimal path to change.

\section{Most Reliable Communication Channel}
If, for every edge weight $w$, we replace $w$ with $w' = -\log(w)$, we can then
run Djikstra's algorithm on the graph to find the most reliable communication
channel. This works due to the fact that the for any real in $0 \leq r \leq 1$,
the closer $r$ is to $0$ the more negative $\log(r)$ will be, and thus
multiplying by $-1$ will give the most reliable paths the smallest values from
$0 \leq w' \leq \infty$.

\section{Minimum Spanning Tree}
Assume we have a graph $G$ with minimal spanning tree $T$. If we cut $T$ between
two vertices $x$ and $y$, we get two new trees $T_1$ and $T_2$. If we consider
the edges in $G$ that join $T_1$ and $T_2$ --- the edges crossing the cut
--- and find the lightest edge, that edge MUST be $x, y$ or else $T$ is not
a minimum spanning tree.

Thus, assume every cut of $G$ to have a unique light edge $x, y$ and $T$ to be
an empty tree. For every cut, we add $x, y$ to $T$ if $x$ or $y$ or both are
not in $T$. This will guarantee that $T$ is a minimal spanning tree, and thus
$T$ exists.

Contradiction for the converse: Suppose we have a graph with vertices $A, B, C$ 
and edges $w(A, B) = 1, w(B, C) = 1, w(A, C) = 2$. This graph has a minimal 
spanning tree containing $(A, B), (B, C)$. However, the cut that cuts the graph
into $B$ and $A, C$ does not have a unique light edge crossing the cut.

\section{Edge Disjoint Trees}
Let $S$ be the tree of shortest paths from $s$ to every other vertex in a graph
$G$. Let $T$ be the minimum spanning tree of $G$. Since $T$ is a minimum
spanning tree, there must be some edge $e$ with minimal weight connecting $s$
to another vertex in $T$. Since $S$ is a tree of shortest paths, the edge $e$
must be contained in the set of edges connecting $s$ to it's direct descendants
in $S$. Thus, it is not possible for there to be an $S$ and a $G$ which do not
contain any edges.


\section{Connected Components}
In an undirected graph, a connected component is any maximal set of vertices
with paths between them, i.e., there can only be more than one connected
component if there are two vertices with no edge between them.

Thus, if you have an undirected graph with $n$ vertices, with $k$ connected
components, drawing an edge from any connected component to any other will
increase the edge count e by 1 and decrease the number of connected components 
by 1. Thus, if $n - k \geq e$, then $n - (k - 1) \geq e + 1$ and the inequality
holds since we have balanced it via subtracting one from the left and adding 
one to the right.

\end{document}
